\subsection{\texttt{bingle\_output.py}}\label{code:bingle_output}

\begin{verbatim}
#bingle_output.py: A two-particle collision simulation 
#                  with ASCII data output
#	Author: D. Weatherley
#	Date: 15 May 2007
#	Organisation: ESSCC, University of Queensland
#	(C) All rights reserved, 2007.
#
#


#provide forward compatibility for Python 2 interpreters
from __future__ import print_function

#import the appropriate ESyS-Particle modules:
from esys.lsm import *
from esys.lsm.util import Vec3, BoundingBox

#instantiate a simulation object 
#and initialise the neighbour search algorithm:
sim = LsmMpi(numWorkerProcesses=1, mpiDimList=[1,1,1])
sim.initNeighbourSearch(
   particleType="NRotSphere",
   gridSpacing=2.5,
   verletDist=0.5
)

#set the number of timesteps and timestep increment:
sim.setNumTimeSteps(10000)
sim.setTimeStepSize(0.001)

#specify the spatial domain for the simulation:
domain = BoundingBox(Vec3(-20,-20,-20), Vec3(20,20,20))
sim.setSpatialDomain(domain)

#add the first particle to the domain:
particle=NRotSphere(id=0, posn=Vec3(-5,5,-5), radius=1.0, mass=1.0)
particle.setLinearVelocity(Vec3(1.0,-1.0,1.0))
sim.createParticle(particle)

#add the second particle to the domain:
particle=NRotSphere(id=1, posn=Vec3(5,5,5), radius=1.5, mass=2.0)
particle.setLinearVelocity(Vec3(-1.0,-1.0,-1.0))
sim.createParticle(particle)

#specify the type of interactions between colliding particles:
sim.createInteractionGroup(
   NRotElasticPrms(
      name = "elastic_repulsion",
      normalK = 10000.0,
      scaling = True
   )
)

#compute the specified number of timesteps:
N_max = sim.getNumTimeSteps()
n=0
while (n < N_max):
   #compute a single timestep:
   sim.runTimeStep()

   # print the particle position to stdout every 100 timesteps:
   if (n%100==0):
      particles = sim.getParticleList()
      p1 = particles[0].getPosn()
      p2 = particles[1].getPosn()
      print(n,p1,p2)

   # update the total number of timesteps computed (n):
   n += 1

#Exit the simulation.
sim.exit()
\end{verbatim}
