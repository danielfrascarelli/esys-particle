\subsection{\texttt{bingle\_Runnable.py}}\label{code:bingle_Runnable}

\begin{verbatim}
#bingle_Runnable.py: A two-particle collision simulation 
#                    with visualisation via a Runnable 
#	Author: D. Weatherley
#	Date: 17 May 2007
#	Organisation: ESSCC, University of Queensland
#	(C) All rights reserved, 2007.
#
#
#import the appropriate ESyS-Particle modules:
from esys.lsm import *
from esys.lsm.util import Vec3, BoundingBox
from POVsnaps import POVsnaps

#instantiate a simulation object 
#and initialise the neighbour search algorithm:
sim = LsmMpi(numWorkerProcesses=1, mpiDimList=[1,1,1])
sim.initNeighbourSearch(
   particleType="NRotSphere",
   gridSpacing=2.5,
   verletDist=0.5
)

#set the number of timesteps and timestep increment:
sim.setNumTimeSteps(10000)
sim.setTimeStepSize(0.001)

#specify the spatial domain for the simulation:
domain = BoundingBox(Vec3(-20,-20,-20), Vec3(20,20,20))
sim.setSpatialDomain(domain)

#add the first particle to the domain:
particle=NRotSphere(id=0, posn=Vec3(-5,5,-5), radius=1.0, mass=1.0)
particle.setLinearVelocity(Vec3(1.0,-1.0,1.0))
sim.createParticle(particle)

#add the second particle to the domain:
particle=NRotSphere(id=1, posn=Vec3(5,5,5), radius=1.5, mass=2.0)
particle.setLinearVelocity(Vec3(-1.0,-1.0,-1.0))
sim.createParticle(particle)

#specify the type of interactions between colliding particles:
sim.createInteractionGroup(
   NRotElasticPrms(
      name = "elastic_repulsion",
      normalK = 10000.0,
      scaling = True
   )
)

#add a POVsnaps Runnable for taking images 
#of the particles every 100 timesteps:
povcam = POVsnaps(sim=sim, interval=100)
sim.addPostTimeStepRunnable(povcam)

#execute the simulation:
sim.run()
\end{verbatim}

