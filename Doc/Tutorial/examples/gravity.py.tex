\subsection{\texttt{gravity.py}}\label{code:gravity}

\begin{verbatim}
#gravity.py: A simple bouncing ball simulation using ESyS-Particle
#	Author: D. Weatherley
#	Date: 15 May 2007
#	Organisation: ESSCC, University of Queensland
#	(C) All rights reserved, 2007.
#
#
#import the appropriate ESyS-Particle modules:
from esys.lsm import *
from esys.lsm.util import Vec3, BoundingBox
from POVsnaps import POVsnaps

#instantiate a simulation object 
#and initialise the neighbour search algorithm:
sim = LsmMpi(numWorkerProcesses=1, mpiDimList=[1,1,1])
sim.initNeighbourSearch(
   particleType="NRotSphere",
   gridSpacing=2.5,
   verletDist=0.5
)

#set the number of timesteps and timestep increment:
sim.setNumTimeSteps(20000)
sim.setTimeStepSize(0.001)

#specify the spatial domain for the simulation:
domain = BoundingBox(Vec3(-20,-20,-20), Vec3(20,20,20))
sim.setSpatialDomain(domain)

#add a particle to the domain:
particle=NRotSphere(id=0, posn=Vec3(0,5,0), radius=1.75, mass=1.8)
particle.setLinearVelocity(Vec3(1.0,10.0,1.0))
sim.createParticle(particle)

#initialise gravity in the domain:
sim.createInteractionGroup(
   GravityPrms(name="earth-gravity", acceleration=Vec3(0,-9.81,0))
)

#add a horizontal wall to act as a floor on which to bounce particles:
sim.createWall(
   name="floor",
   posn=Vec3(0,-10,0),
   normal=Vec3(0,1,0)
)

#specify the type of interactions between wall and particles:
sim.createInteractionGroup(
   NRotElasticWallPrms(
      name = "elasticWall",
      wallName = "floor",
      normalK = 10000.0
   )
)

#add local viscosity to simulate air resistance:
sim.createInteractionGroup(
    LinDampingPrms(
        name="linDamping",
        viscosity=0.1,
        maxIterations=100
    )
)

#add a POVsnaps Runnable:
povcam = POVsnaps(sim=sim, interval=100)
povcam.configure()
sim.addPostTimeStepRunnable(povcam)

#execute the simulation
sim.run()
\end{verbatim}

